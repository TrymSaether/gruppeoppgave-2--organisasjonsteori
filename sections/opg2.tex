\section{Oppgave 2}
\subsection{Arbeidsgiver og motivasjon}
Twoday er \enquote{(...) en internasjonal leverandør av forretningskritiske IT-løsninger i Norden.}
\parencite{Twoday.no}. Twoday har 2700 ansatte og over 8000 kunder. De jobber med: Data og AI, Software Enginering og Digital Experiences. Selskapet tilbyr variasjon i mulige arbeidsoppgaver, og nettsiden viser at de
verdsetter både generalister og spesialister. Tilpasningsevne og drivkraft er kjerneverdier hos twoday. Twoday er
en arbeidsgiver vi ser på som attraktiv, og vi vil videre diskutere hvordan deres fremstilling som seg selv som 
arbeidsgiver er motiverende på oss med grunnlag i pensumlitteraturen. 

Teorien om psykologiske jobbkrav vektlegger læring og variasjon presenteres som krav for å være tilfreds på jobb \parencite[][s.120]{Teknologiledelse}. 
Kjerneverdiene knyttet til tilpasningsevne og drivkraft er i lys av dette svært viktige for å skape motivasjon hos ansatte. Dette er også motiverende for 
oss da vi ønsker en arbeidsgiver hvor vi kan fortsette å utvikle oss og lære slik vi er vant til fra studiehverdagen.
Satsing på personlig utvikling og muligheten til å bestemme selv står sentralt hos twoday: \enquote{Du har muligheten
til å påvirke egen hverdag, og får en bratt læringskurve og mange utfordringer. Vår ambisjon er å
skape stolte øyeblikk(...).} \parencite["Graduateprogrammet"]{Twoday.no}. I tillegg til å vektlegge læring, ser
vi her at twoday vektlegger autonomi. Som beskrevet i Self Determination Theory\parencite[][s.121]{Teknologiledelse},
autonomi og å jobbe av egen fri vilje sentralt for å skape indre motivasjon hos mennesker. Læring,
variasjon og å bestemme over eget arbeid er sentralt for selvrealisering-toppen av Maslows
behovspyramide \parencite[][s.116]{Teknologiledelse}.

Twoday prioriterer mennesker, samhold og bærekraft:  
«Vi bryr oss om våre ansatte, kunder og samfunnet …» \parencite["Om oss"]{Twoday.no}.  
Slike verdier møter både sosiale behov i Maslows pyramide og kravet “sammenheng unngår
fremmedgjøring” fra teorien om psykologiske jobbkrav \parencite[][s.120]{Teknologiledelse}. i jobbkrav-teorien \parencite[][s. 120]{Teknologiledelse}.  
I tillegg ønsker selskapet å gjøre hverdagen enklere for folk \parencite["Graduateprogrammet"]{Twoday.no}, noe som appellerer til vår motivasjon om å bidra til noe større.  



Med vekt på motiverende faktorer som sosialt samhold, selvutvikling, læring, mestring og utfordringer er derfor
twoday en motiverende arbeidsgiver. Flere sentrale motiverende faktorer fra litteraturen, som vi også subjektivt verdsetter
er tydelig i fokus hos dem, noe som gjør dem til en motiverende og interessant arbeidsgiver, \parencite["Graduateprogrammet"]{Twoday.no}. Dette er motiverende for oss, da vi ønsker å delta i noe
større, og utgjøre en forskjell med den jobben vi gjør.



