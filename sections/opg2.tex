\section{Oppgave 2}
\subsection{Valg av arbeidsgiver}
Twoday er “(...) en internasjonal leverandør av forretningskritiske IT-løsninger i Norden.” 
(twoday.no). Ifølge nettsiden har twoday mer enn 2700 ansatte og over 8000 kunder. Innenfor IT 
tilbyr selskapet en variasjon av områder å jobbe med som blant annet: Data og AI, Software Engeeering 
og Digital Experiences. Selskapet tilbyr variasjon i mulige arbeidsoppgaver, og nettsiden viser at de
 verdsetter både generalister og spesialister. Tilpasningsevne og drivkraft er kjerneverdier hos twoday.
   Dette viser at det er gode muligheter for å utvikle seg, lære og prøve nye ting som ansatt hos dem, 
   noe vi ser på som svært motiverende. Dette stemmer overens med teorien psykologiske jobbkrav hvor 
   læring og variasjon presenteres som krav for å være tilfreds på jobb (s.139 i pdf). Satsing på 
   personlig utvikling og muligheten til å bestemme selv står sentralt hos twoday: “Du har muligheten 
   til å påvirke egen hverdag, og får en bratt læringskurve og mange utfordringer. Vår ambisjon er å
    skape stolte øyeblikk(...).” (twoday.no, Graduateprogrammet). I tillegg til å vektlegge læring, ser
    vi her at twoday har vektlegger autonomi. Som beskrevet i Self Determination Theory (s. 140 pdf), 
    autonomi og å jobbe av egen fri vilje sentralt for å skape indre motivasjon hos mennesker. Læring, 
    variasjon og å bestemme over eget arbeid er sentralt for selvrealisering-toppen av Marlows 
    behovspyramide (s.135).


\subsection{Motivasjon}
Twoday prioriterer mennesker, samhold, og bærekraft. Dette fanges opp verdien hjerte:
 “Vi bryr oss om våre ansatte, våre kunder og samfunnet. Vi er en varm virksomhet som er fylt 
 med mennesker som har det gøy og ler sammen. Vi tar imot folk slik de er, vi samarbeider og vi 
 blomstrer sammen.” (twoday.no, Om oss). Vi motiveres av muligheten til å kunne knytte sosiale bånd 
 og utvikle oss sammen med likesinnede. Sosiale behov er sentrale i Marlows behovspyramide, å dekke 
 dette behovet p arbeidsplassen kan være med på å øke trivsel generelt. Vi motiveres også av å delta 
 i noe større og se fruktene av vårt arbeid. Dette stemmer overens med kravet: “sammenheng unngår 
 fremmedgjøring” fra teorien om psykologiske jobbkrav (s.139). I tillegg til å ha fokus på samhold 
 mellom de ansatte er samfunnsnytte og det å gjøre hverdagen enklere for folk viktig hos 
 twoday(twoday.no, Graduateprogrammet). Dette er motiverende for oss, da vi ønsker å delta i noe 
 større, og utgjøre en forskjell med den jobben vi gjør.  