\section{Oppgave 2}
\subsection{Valg av arbeidsgiver}
Twoday er \enquote{(...) en internasjonal leverandør av forretningskritiske IT-løsninger i Norden.}
\cite{twoday.no}. Ifølge nettsiden har twoday mer enn 2700 ansatte og over 8000 kunder. Innenfor IT
tilbyr selskapet en variasjon av områder å jobbe med som blant annet: Data og AI, Software Engieering
og Digital Experiences. Selskapet tilbyr variasjon i mulige arbeidsoppgaver, og nettsiden viser at de
verdsetter både generalister og spesialister. Tilpasningsevne og drivkraft er kjerneverdier hos twoday. Twoday er
en arbeidsgiver vi ser på som atraktiv, og vi vil videre diskutere hvordan deres fremstilling som seg selv som 
arbeidsgiver er motiverende på oss med grunnlag i pensumliteraturen. 

Teorien om psykologiske jobbkrav vektlegger læring og variasjon presenteres som krav for å være tilfreds på jobb \parencite[][s.120]{Teknologiledelse}. 
Kjerneverdiene knyttet til tilpasningsevne og drivkraft er i lys av dette svært viktige for å skape motivasjon hos ansatte. Dette er også motiverende for 
oss da vi ønsker en arbeidsgiver hvor vi kan fotstte å utvikle oss og lære slik vi er vant til fra studiehverdagen.
Satsing på personlig utvikling og muligheten til å bestemme selv står sentralt hos twoday: \enquote{Du har muligheten
til å påvirke egen hverdag, og får en bratt læringskurve og mange utfordringer. Vår ambisjon er å
skape stolte øyeblikk(...).} \parencite["Graduateprogrammet"]{twoday.no}. I tillegg til å vektlegge læring, ser
vi her at twoday vektlegger autonomi. Som beskrevet i Self Determination Theory\parencite[][s.121]{Teknologiledelse},
autonomi og å jobbe av egen fri vilje sentralt for å skape indre motivasjon hos mennesker. Læring,
variasjon og å bestemme over eget arbeid er sentralt for selvrealisering-toppen av Mslows
behovspyramide \parencite[][s.116]{Teknologiledelse}.

Twoday prioriterer mennesker, samhold, og bærekraft. Dette fanges opp verdien hjerte:
“Vi bryr oss om våre ansatte, våre kunder og samfunnet. Vi er en varm virksomhet som er fylt
med mennesker som har det gøy og ler sammen. Vi tar imot folk slik de er, vi samarbeider og vi
blomstrer sammen.” \parencite["Om oss"]{twoday.no}. Vi motiveres av muligheten til å kunne knytte sosiale bånd
og utvikle oss sammen med likesinnede. Sosiale behov er sentrale i Marlows behovspyramide, å dekke
dette behovet på arbeidsplassen kan være med på å øke trivsel generelt. Vi motiveres også av å delta
i noe større og se fruktene av vårt arbeid. Dette stemmer overens med kravet: “sammenheng unngår
fremmedgjøring” fra teorien om psykologiske jobbkrav \parencite[][s.120]{Teknologiledelse}. I tillegg til å ha fokus på samhold
mellom de ansatte er samfunnsnytte og det å gjøre hverdagen enklere for folk viktig hos
twoday\parencite["Graduateprogrammet"]{twoday.no}. Dette er motiverende for oss, da vi ønsker å delta i noe
større, og utgjøre en forskjell med den jobben vi gjør.

Med vekt på motiverende faktorer som sosialt samhold, selvutvikling, læring, mestring og utfordringer er derfor
twoday en motiverende arbeidsgiver. Flere sentrale motiverende faktorer fra litteraturen, som vi også subjektivt verdsetter
er tydelig i fokus hos dem, noe som gjør dem til en motiverende og intressant arbeidsgiver, \parencite["Graduateprogrammet"]{twoday.no}. Dette er motiverende for oss, da vi ønsker å delta i noe
større, og utgjøre en forskjell med den jobben vi gjør.

Med vekt på motiverende faktorer som sosialt samhold, selvutvikling, læring, mestring og utfordringer er derfor
twoday en motiverende arbeidsgiver. Flere sentrale motiverende faktorer fra litteraturen, som vi også subjektivt verdsetter
er tydelig i fokus hos dem, noe som gjør dem til en motiverende og intressant arbeidsgiver. 



