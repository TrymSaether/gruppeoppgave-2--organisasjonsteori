\subsection{Var Ski-VM en suksess?}
Det finnes ingen konkret, universalt svar på om Ski-VM i Trondheim er en suksess eller ikke.  I likhet med det meste annet er dette en subjektiv opplevelse, og dermed er det viktigere spørsmålet for hvem var det en suksess og for hvem var det ikke? For et prosjekt med størrelsen av Ski-VM er det enormt med aktører og interessenter, derfor velger vi tre som er spesielt interessante, Trondheim kommune, Næringslivet rettet mot Ski-VM og Skiforbundet. Dette er aktører som hadde spesielt mye innflytelse på-, og var spesielt påvirket av resultatet og konsekvensene av arrangementet. For å analysere disse gruppene benyttes en PESTEL-analyse (KILDE I BOK).
 Trondheim kommune var den viktigste interessenten av Ski-VM da de klart hadde mest innflytelse på Ski-VM. Dette er en stor og kompleks gruppe med mange ulike seksjoner som alle har ulike mål og ønsker til Ski-VM, likevel kan vi oppdele kommunens ønsker inn i følgende hovedmål: Økonomisk kontroll, byprofilering og innbyggeropplevelse. (Kilde?)
PESTEL-analyse for Trondheim kommune:

Næringslivet rettet mot Ski-VM var også en sentral interessent. Denne gruppen består primært av hoteller, restauranter, transporttjenester og andre virksomheter som i stor grad hadde sitt daglige virke koblet til Ski-VM, enten direkte gjennom tjenester til publikum og deltakere, eller indirekte gjennom økt aktivitet i byen. For næringslivet var hovedmålet å oppnå økt omsetning, bygge nettverk, og styrke Trondheim som en attraktiv destinasjon for fremtidige arrangementer og turisme. (Kilde?)
PESTEL-analyse for næringslivet:


Skiforbundet er en annen sentral interessent og har hatt stor påvirkning på Ski-VMs utforming og gjennomføring. Skiforbundets hovedmål har vært å sikre gode rammebetingelser for utøvere og arrangement, samt å bygge videre på Norges omdømme som vintersportsnasjon. Forbundet har også et sterkt ønske om å promotere skisporten nasjonalt og internasjonalt, og å sikre rekruttering og økt interesse blant unge.
PESTEL-analyse for Skiforbundet:
