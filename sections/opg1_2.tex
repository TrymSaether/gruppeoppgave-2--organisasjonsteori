\subsection{Var Ski-VM en suksess?}
Det finnes ingen konkret, universalt svar på om Ski-VM i Trondheim er en suksess eller ikke.
I likhet med det meste annet er dette subjektivt, og dermed er det viktigere spørsmålet for hvem
var det en suksess og for hvem var det ikke? For et prosjekt med størrelsen av Ski-VM er det
enormt med aktører og interessenter, derfor velger vi tre som er spesielt interessante,
Trondheim kommune, næringslivet rettet mot Ski-VM og Skiforbundet. Dette er aktører som hadde
spesielt mye innflytelse på-, og var spesielt påvirket av resultatet og konsekvensene av
arrangementet. For å analysere disse gruppene benyttes en PESTEL-analyse\parencite[][s.219]{Teknologiledelse}.

\subsubsection{Trondheim kommune}
Trondheim kommune var den viktigste interessenten av Ski-VM da de klart hadde mest
innflytelse på Ski-VM. Dette er en stor og kompleks gruppe med mange ulike seksjoner som
alle har ulike mål og ønsker til Ski-VM, likevel kan kommunens ønsker inndeles i
følgende hovedmål: P - politisk gevinst, E - økonomisk bærekraft og E - miljømessig bærekraft og troverdighet.

\paragraph{P - Politisk gevinst:} VM styrket Trondheims nasjonale omdømme. Noen dager var det mer enn 40 tusen publikummere i Granåsen\parencite{NRKFolkefest}. Statsministerens støtte fremhevet kommunens gjennomføringsevne\parencite{Trondheim2025Midler}. Arrangementet ble omtalt som mer et skimesterskap, med særlig vekt på frivillighet og inkludering\parencite{Trondheim2025Baerekraft}.  Dette samsvarer med kommunens politikk for sosial inkludering, og bidro til økt lokal stolthet og tillit til kommunens prosjektkapasitet. Undersøkelser viser at majoriteten av de som deltok i undersøkelsen "hadde et positivt totalinntrykk av VM og 69 prosent av dem i aldersgruppen 15-29 år syntes regionen er mer attraktiv etter mesterskapet"\parencite{AftenpostenNyVM}.

\paragraph{E - Økonomisk bærekraft:} Trondheim kommune investerte tungt i Granåsen med omlag 1.2 
milliarder kr satt inn i prosjektet\parencite{NRK12Milliarder}. Disse enorme summene har blitt møtt med skepsis,
og i ettertid er det ennå ikke tydelig om VM har direkte gått i pluss for Trondheim kommune\parencite{AftenpostenNyVM}.
Indirekte kan arrangementet ha vært økonomisk bærekraftig gjennom å stimulere den lokal
økonomien, det er tydelig at det har vært en enormt økt aktivitet i hotell- og
restaurantnæringen\parencite{DagbladetPriser}. Kommunen har siden signalisert at fremtidige mesterskap ikke
kan få like mye finansiering fra Trondheim\parencite{NeaRadioVM}, noe som kan tyde på at VM har hatt mindre økonomisk
gevinst enn forventet. Samtidig forblir mye av investeringen igjen til gevist for Trondheim i form
et oppgradert, varig skianlegg.

\paragraph{E - Miljømessig troverdighet:} Kommunen og arrangørene hadde mål om et klimanøytralt 
mesterskap\parencite{TrondheimKommuneVM}. Blant tiltakene var bruk av utslippsfri energi, grønn 
transport og massiv avfallssortering. Klimagassutslipp ble redusert med 98\% sammenlignet 
med Oslo-VM 2011\parencite{Trondheim2025Kutt}. Granåsen ble bygget med miljøhensyn og skal ha lang levetid. 
Kommunens krav presset frem innovative løsninger og styrket Trondheims miljøprofil nasjonalt og internasjonalt.

\subsubsection{Næringslivet rettet mot Ski-VM}
Næringslivet rettet mot Ski-VM var også en viktig interessent. Denne gruppen består av 
hoteller, restauranter, transporttjenester og andre virksomheter som i stor grad hadde sitt 
daglige virke koblet til Ski-VM, enten direkte gjennom tjenester til publikum og deltakere, 
eller indirekte gjennom økt aktivitet i byen. Spesielt viktig innad i denne gruppen er 
næringslivet som ble skapt utelukkende for å ta fordel av Ski-VM, da matbodene og VM-lavvoen 
i sentrum. For næringslivet anslår vi at hovedmålet var å oppnå: P - bygge nettverk, E - økt omsetning,  og S - styrke Trondheim som en attraktiv destinasjon for fremtidige arrangementer og turisme.

\paragraph{P - Politiske vurderinger:}
Det ble politisk debatt om økonomistyring. Kritikere pekte på høyere kostnader enn planlagt, og mente kommunen burde stilt strengere krav\parencite{nettavisenKritikk}. Arrangørene fremhevet på sin side høy sponsorstøtte og publikumsoppslutning som tegn på suksess\parencite{kom24Sponsorsalg}. VM-landsbyen og lavvoen illustrerer hvordan et mesterskap kan forenes med næringsutvikling og byliv.

\paragraph{E - Økonomiske aspekter:}
Næringslivet spilte en nøkkelrolle under Ski-VM i Trondheim. Samtlige sponsorpakker ble solgt, og mange bedrifter opplevde økt omsetning\parencite{kom24Sponsorsalg}. Lokale butikker og serveringssteder, som Rema 1000 ved Granåsen, melder om salgsboom og høy publikumstilstrømning\parencite{nettavisenRema}. Midtbyen utvidet åpningstidene, og lokale aktører som Eggan Nedre tilbød mat, bar og konserter i en stor lavvo\parencite{midtbyenProgram}. Samtidig opplevde enkelte utfordringer, som butikker som fikk utsikten sperret av utedoer midt i handlegata\parencite{nettavisenToalett} Det var også ulik fordeling av inntektene, hvor billigere, uformelle aktører tok noe av salget fra offisielle VM-stand\parencite{nettavisenRema}. I tillegg hadde været en enorm påvirkning på omsetning av VM-bodene. Flere av aktørene rapporterte om store tap både direkte, i form av matsvinn, og potensiell, i form av tapte kunder\parencite{innherredTragedie}.

\paragraph{S - Sosiale effekter:}
Lavvoen og VM-landsbyen ble sentrale for folkefesten i byen, med DJ-er, konserter og 
storskjermer. Enkelte kvelder var det kø for å komme inn i lavvoen, noe som tyder på stor 
interesse. Over 200 tusen besøkte arrangementene på Torvet, med opptil 27 tusen på én kveld\parencite{wikipediaSkiVM}. 
Tilbudene var inkluderende og gratis, og arrangement som “Drømmedagen” for skoleelever bidro til 
sosial bærekraft og lokal stolthet\parencite{wikipediaSkiVM}.

\subsubsection{Norges Skiforbund som interessent}
Skiforbundet var en sentral interessent med stor påvirkning på Ski-VM. Som hovedeier (60\%) i arrangørselskapet og øverste sportslige myndighet hadde NSF en todelt rolle: strategisk prosjekteier med ansvar for nasjonale mål og kommersiell aktør som skulle sikre økonomisk overskudd\parencite{ProffSkiVM2025}.

\paragraph{Mål og resultater}
NSF hadde fire hovedmål: økonomisk overskudd etter flere år med underskudd\parencite{Adresseavisen}, norske medaljer som styrket merkevaren, bærekraft som viser samfunnsansvar\parencite{TrondheimKommuneVM}, og omdømmebygging for rekruttering\parencite{OsloVM}. VM leverte rekordpublikum, høy TV-rating og norsk medaljedominans. Bærekraftsprofilen traff unge målgrupper, mens økonomisk ble det et moderat overskudd - lavere enn håpet men viktig for forbundet\parencite{Trondheim2025Portal,Adresseavisen}.

\paragraph{Strategiske konsekvenser}
Erfaringene fra Trondheim vil prege forbundets strategi gjennom økt satsing på bærekraft og teknologi som konkurransefortrinn, større vilje til prosjekteierskap i partnerskap med andre aktører, og tydeligere publikums- og rekrutteringsstrategi for å omsette VM-effekten til flere aktive medlemmer.

\paragraph{PESTEL-analyse}

\begin{table}[H]
    \centering
    \begin{tabular}{@{}p{2.7cm}p{10.2cm}@{}}
        \toprule                                                                                                                                                                                            \\ \midrule
        \textbf{P}olitiske     & Statlig og kommunal finansiering avhang av at arrangementet leverte brede samfunnsgevinster \parencite{TrondheimKommuneVM}.                                                     \\
        \textbf{E}konomiske    & 93~\% av billettene ble solgt nasjonalt; samtidig presset valutarisiko og økte anleggskostnader resultatmarginen \parencite{AdressaKjopefest}.                                  \\
        \textbf{S}osiale       & 230\,000 stadiontilskuere og 3\,800 frivillige skapte folkefest, men synkende klubbmedlemskap viser at grasroteffekten ikke er realisert ennå \parencite{Adresseavisen,OsloVM}. \\
        \textbf{T}eknologiske  & Prosjektet \textit{Snow for the Future} med automatisert og energieffektiv snøproduksjon gav et eksportbart kompetansefortrinn\parencite{Trondheim2025Sustainability}.          \\
        \textbf{E}nvironmental & Klimatiltak som kortreist mat og 10\,000 nyplantede trær reduserte kritikk, men flyreiser fra publikum forble et omdømmerisiko.                                            \\
        \textbf{L}ovmessige    & FIS-reglement, antidopingkrav og offentlige anskaffelsesregler krevde profesjonell prosjektstyring og åpenhet.                                                             \\ \bottomrule
    \end{tabular}
\end{table}

\subsubsection{Suksessvurdering}
Basert på PESTEL-analysene av de tre hovedinteressentene, fremstår Ski-VM i Trondheim som en betinget suksess med varierende resultater avhengig av perspektiv. For Trondheim kommune var mesterskapet en politisk og miljømessig suksess med styrket byomdømme og 98\% reduserte klimagassutslipp\parencite{Trondheim2025Kutt}, men økonomisk mer nyansert med betydelige investeringer (1,2 milliarder kr) som primært gir langsiktig verdi gjennom varige anlegg. Næringslivet opplevde generelt økt omsetning med utsolgte sponsorpakker og over 200 000 besøkende i sentrum\parencite{wikipediaSkiVM}, dog med ujevn inntektsfordeling og væravhengige tap for enkelte aktører\parencite{innherredTragedie}. Norges Skiforbund oppnådde viktige gevinster gjennom sportslig suksess, rekordpublikum, høy TV-rating og et moderat økonomisk overskudd\parencite{Adresseavisen}. Mesterskapet lyktes særlig i å balansere tradisjon og fornyelse ved å ivareta skiidrettens kulturelle betydning samtidig som det demonstrerte innovasjon innen bærekraft og teknologi, noe som kan tjene som modell for fremtidige idrettsarrangementer.
