\subsection{Var Ski-VM en suksess?}
Det finnes ingen konkret, universalt svar på om Ski-VM i Trondheim er en suksess eller ikke.  I likhet med det meste annet er dette subjektivt, og dermed er det viktigere spørsmålet for hvem var det en suksess og for hvem var det ikke? For et prosjekt med størrelsen av Ski-VM er det enormt med aktører og interessenter, derfor velger vi tre som er spesielt interessante, Trondheim kommune, Næringslivet rettet mot Ski-VM og Skiforbundet. Dette er aktører som hadde spesielt mye innflytelse på-, og var spesielt påvirket av resultatet og konsekvensene av arrangementet. For å analysere disse gruppene benyttes en PESTEL-analyse (KILDE I BOK).

Trondheim kommune var den viktigste interessenten av Ski-VM da de klart hadde mest innflytelse på Ski-VM. Dette er en stor og kompleks gruppe med mange ulike seksjoner som alle har ulike mål og ønsker til Ski-VM, likevel kan vi oppdele kommunens ønsker inn i følgende hovedmål: Økonomisk kontroll, byprofilering og innbyggeropplevelse. (Kilde?)

PESTEL-analyse for Trondheim kommune:

Næringslivet rettet mot Ski-VM var også en viktig interessent. Denne gruppen består av hoteller, restauranter, transporttjenester og andre virksomheter som i stor grad hadde sitt daglige virke koblet til Ski-VM, enten direkte gjennom tjenester til publikum og deltakere, eller indirekte gjennom økt aktivitet i byen. Spesielt viktig innad i denne gruppen er næringslivet som ble skapt utelukkende for å ta fordel av Ski-VM, da matbodene og VM-lavvoen i sentrum. For næringslivet var hovedmålet å oppnå økt omsetning, bygge nettverk, og styrke Trondheim som en attraktiv destinasjon for fremtidige arrangementer og turisme. (Kilde?)

PESTEL-analyse for næringslivet:


Skiforbundet er en annen sentral interessent og har hatt stor påvirkning på Ski-VMs utforming og gjennomføring. Skiforbundets hovedmål har vært å sikre gode rammebetingelser for utøvere og arrangement, og å bygge videre på Norges omdømme som vintersportsnasjon. Forbundet har i tillegg et sterkt ønske om å promotere skisporten nasjonalt og internasjonalt, og å sikre rekruttering og økt interesse blant unge.

PESTEL-analyse for Skiforbundet:

\subsubsection{Norges Skiforbund som interessent}

Som hovedeier (60 \%) i arrangørselskapet Ski-VM Trondheim 2025 AS og øverste sportslige myndighet hadde Norges Skiforbund (NSF) en todelt rolle: strategisk prosjekteier med ansvar for nasjonale mål -- rekruttering, prestasjoner og omdømme -- og kommersiell aktør som skulle sikre et økonomisk overskudd til videre drift og talentutvikling\cite{ProffSkiVM2025}.

\paragraph{Hovedinteresser}

\begin{itemize}
    \item \textbf{Økonomi} -- overskudd som kan snu flere år med røde tall og medlemstap\cite{Adresseavisen}
    \item \textbf{Sportslig prestasjon} -- norske medaljer som styrker merkevaren Langrenn Norge.
    \item \textbf{Bærekraftig arrangement} som viser forbundets samfunnsansvar \cite{TrondheimKommune}.
    \item \textbf{Omdømmebygging} mot frivillige, ungdom og sponsorer -- VM som en del av forbundets rekrutteringsstrategi \cite{OsloVM}.
\end{itemize}

\paragraph{PESTEL-analyse}

\begin{table}[H]
    \centering
    \begin{tabular}{@{}p{2.7cm}p{10.2cm}@{}}
        \toprule
        \textbf{Faktor}        & \textbf{Relevans for NSF}                                                                                                                                                  \\ \midrule
        \textbf{P}olitiske     & Statlig og kommunal finansiering avhang av at arrangementet leverte brede samfunnsgevinster \cite{TrondheimKommune}.                                                       \\
        \textbf{E}konomiske    & 93~\% av billettene ble solgt nasjonalt; samtidig presset valutarisiko og økte anleggskostnader resultatmarginen \cite{AdressaKjopefest}.                                  \\
        \textbf{S}osiale       & 230\,000 stadiontilskuere og 3\,800 frivillige skapte folkefest, men synkende klubbmedlemskap viser at grasroteffekten ikke er realisert ennå \cite{Adresseavisen,OsloVM}. \\
        \textbf{T}eknologiske  & Prosjektet \textit{Snow for the Future} med automatisert og energieffektiv snøproduksjon gav et eksportbart kompetansefortrinn\cite{Trondheim2025Sustainability}.          \\
        \textbf{E}nvironmental & Klimatiltak som kortreist mat og 10\,000 nyplantede trær reduserte kritikk, men flyreiser fra publikum forble et omdømmerisiko.                                            \\
        \textbf{L}ovmessige    & FIS-reglement, antidopingkrav og offentlige anskaffelsesregler krevde profesjonell prosjektstyring og åpenhet.                                                             \\ \bottomrule
    \end{tabular}
\end{table}

\paragraph{Påvirkning og resultater}
Fra et interessent-perspektiv (pensum kap. 9) er suksess mer enn «on time/on budget»; en eier må oppleve \textit{impact success} -- varige gevinster. Ski-VM leverte rekordpublikum, høy TV-rating og norsk medaljedominans, noe som styrket NSF både sportslig og omdømmemessig.
Bærekraftsprofilen traff unge målgrupper og kan hjelpe forbundet i kampen om frivillige og sponsorer.
Økonomisk ble det rapportert et moderat pluss -- lavere enn håpet, men viktige inntekter vil fortsatt komme fra etter-arrangement og markedsrettigheter \cite{Trondheim2025Portal,Adresseavisen}.

\paragraph{Suksessvurdering}
Med tydelige gevinster på tre av fire hovedinteresser kan Ski-VM regnes som en betinget suksess for NSF. Den største svakheten er at medlemstallene foreløpig ikke viser vekst -- et signal om at sosiale effekter må følges opp gjennom bredere grasrot-initiativ og lavterskelarenaer.

\paragraph{Strategisk refleksjon}
Erfaringene fra Trondheim vil trolig prege Skiforbundets strategi:
\begin{itemize}
    \item Økt satsing på bærekraft og teknologi som konkurransefortrinn internasjonalt.
    \item Større vilje til å ta projekteierrollen i partnerskap med kommuner og næringsliv.
    \item En tydeligere publikums- og rekrutteringsstrategi mellom mesterskapene for å omsette VM-effekten til flere aktive medlemmer.
\end{itemize}

Dermed har Ski-VM ikke bare vært en målestokk på NSF sin prosjektkompetanse -- det har også gitt retning for forbundets posisjon i norsk og internasjonal skisport de neste årene.