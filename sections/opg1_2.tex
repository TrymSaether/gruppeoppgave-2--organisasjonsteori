\subsection{Var Ski-VM en suksess?}
Det finnes ingen konkret, universalt svar på om Ski-VM i Trondheim er en suksess eller ikke.
I likhet med det meste annet er dette subjektivt, og dermed er det viktigere spørsmålet for hvem
var det en suksess og for hvem var det ikke? For et prosjekt med størrelsen av Ski-VM er det
enormt med aktører og interessenter, derfor velger vi tre som er spesielt interessante,
Trondheim kommune, Næringslivet rettet mot Ski-VM og Skiforbundet. Dette er aktører som hadde
spesielt mye innflytelse på-, og var spesielt påvirket av resultatet og konsekvensene av
arrangementet. For å analysere disse gruppene benyttes en PESTEL-analyse(\cite{Teknologiledelse} s.219).

\subsubsection{Trondheim kommune}
Trondheim kommune var den viktigste interessenten av Ski-VM da de klart hadde mest
innflytelse på Ski-VM. Dette er en stor og kompleks gruppe med mange ulike seksjoner som
alle har ulike mål og ønsker til Ski-VM, likevel kan kommunens ønsker inndeles i
følgende hovedmål: P - Politisk gevins, E - Økonomisk bærekraft og E - Miljømessig bærekraft og troverdighet.


\paragraph{P - Politisk gevinst:} VM styrket Trondheims nasjonale omdømme. Noen dager var det mer enn 40 tusen publikummere i 
Granåsen\cite{NRKFolkefest}. Statsministerens støtte fremhevet kommunens 
gjennomføringsevne\cite{Trondheim2025Midler}. 
Arrangementet ble omtalt som mer et skimesterskap, med særlig vekt på frivillighet og 
inkludering\cite{Trondheim2025Baerekraft}.  Dette samsvarer med kommunens politikk for sosial inkludering, 
og bidro til økt lokal stolthet og tillit til kommunens prosjektkapasitet. Undersøkelser viser
at majoriteten av de som deltok i undersøkelsen "hadde et positivt totalinntrykk av VM og 69 prosent
av dem i aldersgruppen 15-29 år syntes regionen er mer attraktiv etter mesterskapet"\cite{AftenpostenNyVM}.

\paragraph{E - Økonomisk bærekraft:} Trondheim kommune investerte tungt i Granåsen med omlag 1.2 
milliarder kr satt inn i prosjektet\cite{NRK12Milliarder}. Disse enorme summene har blitt møtt med skepsis,
og i ettertid er det ennå ikke tydelig om VM har direkte gått i pluss for Trondheim kommune\cite{AftenpostenNyVM}.
Indirekte kan arrangementet ha vært økonomisk bærekraftig gjennom å stimulere den lokal
økonomien, det er tydelig at det har vært en enormt økt aktivitet i hotell- og
restaurantnæringen\cite{DagbladetPriser}. Kommunen har siden signalisert at fremtidige mesterskap ikke
kan få like mye finansiering fra Trondheim\cite{NeaRadioVM}, noe som kan tyde på at VM har hatt mindre økonomisk
gevinst enn forventet. Samtidig forblir mye av investeringen igjen til gevist for Trondheim i form
et oppgradert, varig skianlegg.

\paragraph{E - Miljømessig troverdighet:} Kommunen og arrangørene hadde mål om et klimanøytralt 
mesterskap\cite{TrondheimKommuneVM}. Blant tiltakene var bruk av utslippsfri energi, grønn 
transport og massiv avfallssortering. Klimagassutslipp ble redusert med 98\% sammenlignet 
med Oslo-VM 2011\cite{Trondheim2025Kutt}. Granåsen ble bygget med miljøhensyn og skal ha lang levetid. 


\subsubsection{Næringslivet rettet mot Ski-VM}
En sentral del av næringslivet rundt Ski-VM i Trondheim var knyttet til arrangementene i sentrum, 
spesielt VM-landsbyen og VM-lavvoen. Disse tiltakene skulle ikke bare skape økonomisk aktivitet, 
men også bidra til folkefesten og markedsføre byen som arrangementsdestinasjon. Ved å analysere 
disse initiativene gjennom utvalgte dimensjoner, P - politiske forhold, E - økonomiske aspekter og 
S - sosiale effekter, får vi et tydeligere bilde av hvordan næringslivet ble påvirket og hvilken 
rolle det spilte i gjennomføringen og opplevelsen av mesterskapet.

\paragraph{P - Politiske vurderinger:}
Det ble politisk debatt om økonomistyring. Kritikere pekte på høyere kostnader enn planlagt, 
og mente kommunen burde stilt strengere krav\parencite{nettavisenKritikk}. Arrangørene fremhevet på sin side høy sponsorstøtte og 
publikumsoppslutning som tegn på suksess\parencite{kom24Sponsorsalg}. VM-landsbyen og lavvoen illustrerer hvordan 
et mesterskap kan forenes med næringsutvikling og byliv.


\paragraph{E - Økonomiske aspekter:}
Næringslivet spilte en nøkkelrolle under Ski-VM i Trondheim. Samtlige sponsorpakker ble solgt, og 
mange bedrifter opplevde økt omsetning\parencite{kom24Sponsorsalg}. Lokale butikker og serveringssteder, som Rema 1000 
ved Granåsen, melder om salgsboom og høy publikumstilstrømning\parencite{nettavisenRema}. Midtbyen utvidet åpningstidene,
og lokale aktører som Eggan Nedre tilbød mat, bar og konserter i en stor lavvo\parencite{midtbyenProgram}. 
Samtidig opplevde enkelte utfordringer, som butikker som fikk utsikten sperret av 
utedoer midt i handlegata\parencite{nettavisenToalett} Det var også ulik fordeling av inntektene, hvor billigere, 
uformelle aktører tok noe av salget fra offisielle VM-stands(nettavisenRema). I tillegg hadde været en enorm påvirkning
på omsetning av VM-bodene. Flere av aktørene rapporterte om store tap både direkte, i form av matsvinn, og
potensiell, i form av tapte kunder\parencite{innherredTragedie}.

\paragraph{S - Sosiale effekter:}
Lavvoen og VM-landsbyen ble sentrale for folkefesten i byen, med DJ-er, konserter og 
storskjermer. Enkelte kvelder var det kø for å komme inn i lavvoen, noe som tyder på stor 
interesse. Over 200 tusen besøkte arrangementene på Torvet, med opptil 27 tusen på én kveld\parencite{wikipediaSkiVM}. 
Tilbudene var inkluderende og gratis, og arrangement som “Drømmedagen” for skoleelever bidro til 
sosial bærekraft og lokal stolthet\parencite{wikipediaSkiVM}.

\subsubsection{Norges Skiforbund (NSF)}
Skiforbundet var en annen viktig interessent med betydelig innflytelse på Ski-VMs gjennomføring.
Som majoritetseier (60 \%) i Ski-VM Trondheim 2025 AS og øverste sportslige instans, hadde
NSF et todelt ansvar: De fungerte både som strategisk prosjekteier med fokus
på nasjonale mål (rekruttering, resultater og omdømme) og som en kommersiell aktør som
måtte sikre økonomisk overskudd \cite{ProffSkiVM2025}.

\paragraph{Hovedinteresser}
\begin{itemize}
    \item \textbf{Økonomi} -- skape overskudd for å motvirke flere år med røde tall og medlemstap \cite{Adresseavisen}.
    \item \textbf{Sportslig suksess} -- sikre norske medaljer og styrke Langrenn Norges renommé.
    \item \textbf{Bærekraft} -- bygge troverdighet gjennom et miljøvennlig arrangement \cite{TrondheimKommuneVM}.
    \item \textbf{Omdømmebygging} -- nå ungdom, frivillige og sponsorer for å rekruttere flere aktive medlemmer \cite{OsloVM}.
\end{itemize}

\paragraph{PESTEL-analyse}
\begin{table}[ht]
    \centering
    \begin{tabular}{@{}p{2.7cm}p{10.2cm}@{}}
        \toprule \\ \midrule
        \textbf{P}olitiske     & Statlig og kommunal støtte kom med krav om målbare samfunnsgevinster \cite{TrondheimKommuneVM}. \\
        \textbf{E}konomiske    & 93\,\% av billettene ble solgt, men valutarisiko og høye anleggskostnader svekket marginen \cite{AdressaKjopefest}. \\
        \textbf{S}osiale       & 230\,000 tilskuere og 3\,800 frivillige skapte folkefest, men medlemsveksten uteble \cite{Adresseavisen,OsloVM}. \\
        \textbf{T}eknologiske  & Prosjektet \textit{Snow for the Future} ga automatisert, energieffektiv snøproduksjon med eksportpotensial \cite{Trondheim2025Sustainability}. \\
        \textbf{E}nvironmental & Tiltak som kortreist mat og 10\,000 nyplantede trær reduserte klimakritikk, men flytrafikk fra publikum var fortsatt en omdømmerisiko. \\
        \textbf{L}ovmessige    & FIS-regelverk, antidopingkrav og offentlige anskaffelsesregler krevde profesjonell styring med åpenhet. \\ \bottomrule
    \end{tabular}
\end{table}

\paragraph{Påvirkning og resultater}
Fra et interessentperspektiv må NSF som prosjekteier oppnå varige gevinster (\textit{impact success}). 
Arrangementet leverte solid på flere områder: Publikumsrekorder og høye seertall kombinert med norsk medaljefangst 
styrket forbundets sportslige posisjon og merkevare. Det tydelige bærekraftfokuset appellerte til yngre målgrupper, 
noe som potensielt kan styrke rekrutteringen av både sponsorer og frivillige på lengre sikt.
Økonomisk ga mesterskapet et moderat overskudd initielt, men forventningene til ytterligere inntekter 
fra etterbruk av anlegg og fremtidige arrangementsrettigheter tilsier at det finansielle resultatet 
kan forbedres vesentlig\cite{Trondheim2025Portal,Adresseavisen}.

Erfaringene fra Trondheim vil sannsynligvis påvirke Skiforbundets strategi:
\begin{itemize}
    \item Mer satsing på bærekraft og teknologi som internasjonale konkurransefortrinn.
    \item Styrket projekteierrolle i samarbeid med kommuner og næringsliv.
    \item Bedre publikums- og rekrutteringsstrategi mellom mesterskapene for å bevare VM-effekten og øke andelen aktive i skisporten.
\end{itemize}

\subsubsection{Suksessvurdering}
Med tydelige gevinster på tre av fire hovedområder kan Ski-VM anses som en betinget suksess for NSF.
Svakheten er manglende medlemsvekst, noe som indikerer behov for mer grasrotarbeid og lavterskeltilbud.
Dermed har mesterskapet ikke bare skapt en arena for NSFs prosjektkompetanse, men også pekt ut veien
for neste fase i norsk og internasjonal skisport.
