\subsection{Var Ski-VM en suksess?}
Det finnes ingen konkret, universalt svar på om Ski-VM i Trondheim er en suksess eller ikke.  
I likhet med det meste annet er dette subjektivt, og dermed er det viktigere spørsmålet for hvem 
var det en suksess og for hvem var det ikke? For et prosjekt med størrelsen av Ski-VM er det 
enormt med aktører og interessenter, derfor velger vi tre som er spesielt interessante, 
Trondheim kommune, Næringslivet rettet mot Ski-VM og Skiforbundet. Dette er aktører som hadde 
spesielt mye innflytelse på-, og var spesielt påvirket av resultatet og konsekvensene av 
arrangementet. For å analysere disse gruppene benyttes en PESTEL-analyse(\cite{Teknologiledelse} s.219).

\subsubsection{Trondheim kommune}
Trondheim kommune var den viktigste interessenten av Ski-VM da de klart hadde mest 
innflytelse på Ski-VM. Dette er en stor og kompleks gruppe med mange ulike seksjoner som 
alle har ulike mål og ønsker til Ski-VM, likevel kan kommunens ønsker inndeles i 
følgende hovedmål: P - Politisk gevins, E - Økonomisk bærekraft og E - Miljømessig bærekraft og troverdighet.


\paragraph{P - Politisk gevinst:} VM styrket Trondheims nasjonale omdømme. Mer enn 519 milioner så på 
Ski-VM\cite{AftenpostenSeertall} og noen dager var det mer enn 40 tusen publikummere i 
Granåsen\cite{NRKFolkefest}. Statsministerens støtte fremhevet kommunens 
gjennomføringsevne\cite{Trondheim2025Midler}. 
Arrangementet ble omtalt som mer et skimesterskap, med særlig vekt på frivillighet og 
inkludering\cite{Trondheim2025Baerekraft}.  Dette samsvarer med kommunens politikk for sosial inkludering, 
og bidro til økt lokal stolthet og tillit til kommunens prosjektkapasitet. Undersøkelser viser
at majoriteten av de som deltok i undersøkelsen "hadde et positivt totalinntrykk av VM og 69 prosent 
av dem i aldersgruppen 15-29 år syntes regionen er mer attraktiv etter mesterskapet"\cite{AftenpostenNyVM}.

\paragraph{E - Økonomisk bærekraft:} Trondheim kommune investerte tungt i Granåsen med omlag 1.2 
milliarder kr satt inn i prosjektet\cite{NRK12Milliarder}. Disse enorme summene har blitt møtt med skepsis,
og i ettertid er det ennå ikke tydelig om VM har direkte gått i pluss for Trondheim kommune\cite{AftenpostenNyVM}. 
Indirekte kan arrangementet ha vært økonomisk bærekraftig gjennom å stimulere den lokal 
økonomien, det er tydelig at det har vært en enormt økt aktivitet i hotell- og 
restaurantnæringen\cite{DagbladetPriser}. Kommunen har siden signalisert at fremtidige mesterskap ikke
kan få like mye finansiering fra Trondheim\cite{NeaRadioVM}, noe som kan tyde på at VM har hatt mindre økonomisk 
gevinst enn forventet. Samtidig forblir mye av investeringen igjen til gevist for Trondheim i form 
et oppgradert, varig skianlegg.

\paragraph{E - Miljømessig troverdighet:} Kommunen og arrangørene hadde mål om et klimanøytralt 
mesterskap\cite{TrondheimKommuneVM}. Blant tiltakene var bruk av utslippsfri energi, grønn 
transport og massiv avfallssortering. Klimagassutslipp ble redusert med 98\% sammenlignet 
med Oslo-VM 2011\cite{Trondheim2025Kutt}. Granåsen ble bygget med miljøhensyn og skal ha lang levetid. 


\subsubsection{Næringslivet rettet mot Ski-VM}
En sentral del av næringslivet rundt Ski-VM i Trondheim var knyttet til arrangementene i sentrum, 
spesielt VM-landsbyen og VM-lavvoen. Disse tiltakene skulle ikke bare skape økonomisk aktivitet, 
men også bidra til folkefesten og markedsføre byen som arrangementsdestinasjon. Ved å analysere 
disse initiativene gjennom utvalgte dimensjoner, P - politiske forhold, E - økonomiske aspekter og 
S - sosiale effekter, får vi et tydeligere bilde av hvordan næringslivet ble påvirket og hvilken 
rolle det spilte i gjennomføringen og opplevelsen av mesterskapet.

\paragraph{P - Politiske vurderinger:}
Det ble politisk debatt om økonomistyring. Kritikere pekte på høyere kostnader enn planlagt, 
og mente kommunen burde stilt strengere krav(kilde). Blandt annet ble 3,5 millioner brukt på busser 
og 13 millioner på økt politiberedskap(kilde). Arrangørene fremhevet på sin side høy sponsorstøtte og 
publikumsoppslutning som tegn på suksess(Kilde). VM-landsbyen og lavvoen illustrerer hvordan 
et mesterskap kan forenes med næringsutvikling og byliv.


\paragraph{E - Økonomiske aspekter:}
Næringslivet spilte en nøkkelrolle under Ski-VM i Trondheim. Samtlige sponsorpakker ble solgt, og 
mange bedrifter opplevde økt omsetning. Lokale butikker og serveringssteder, som Rema 1000 
ved Granåsen, melder om salgsboom og høy publikumstilstrømning(Kilde). Midtbyen utvidet åpningstidene,
og lokale aktører som Eggan Nedre tilbød mat, bar og konserter i en stor lavvo(kilde). 
Samtidig opplevde enkelte utfordringer, som butikker som fikk utsikten sperret av 
utedoer midt i handlegata. Det var også ulik fordeling av inntektene, hvor billigere, 
uformelle aktører tok noe av salget fra offisielle VM-stands. I tillegg hadde været en enorm påvirkning
på omsetning av VM-bodene. Flere av aktørene rapporterte om store tap både direkte, i form av matsvinn, og
potensiell, i form av tapte kunder(KIlde).

\paragraph{S - Sosiale effekter:}
Lavvoen og VM-landsbyen ble sentrale for folkefesten i byen, med DJ-er, konserter og 
storskjermer. Enkelte kvelder var det kø for å komme inn i lavvoen, noe som tyder på stor 
interesse(Kilde). Over 200 tusen besøkte arrangementene på Torvet, med opptil 27 tusen på én kveld(kilde). 
Tilbudene var inkluderende og gratis, og arrangement som “Drømmedagen” for skoleelever bidro til 
sosial bærekraft og lokal stolthet(Kilde).


\subsubsection{Norges Skiforbund som interessent}
Skiforbundet er en annen sentral interessent og har hatt stor påvirkning på Ski-VMs utforming 
og gjennomføring. Som hovedeier (60 \%) i arrangørselskapet Ski-VM Trondheim 2025 AS og øverste 
sportslige myndighet hadde Norges Skiforbund (NSF) en todelt rolle: strategisk prosjekteier med 
ansvar for nasjonale mål -- rekruttering, prestasjoner og omdømme -- og kommersiell aktør som 
skulle sikre et økonomisk overskudd til videre drift og talentutvikling\cite{ProffSkiVM2025}.

\paragraph{Hovedinteresser}

\begin{itemize}
    \item \textbf{Økonomi} -- overskudd som kan snu flere år med røde tall og 
    medlemstap\cite{Adresseavisen}

    \item \textbf{Sportslig prestasjon} -- norske medaljer som styrker merkevaren Langrenn Norge.
    
    \item \textbf{Bærekraftig arrangement} som viser forbundets samfunnsansvar \cite{TrondheimKommuneVM}.

    \item \textbf{Omdømmebygging} mot frivillige, ungdom og sponsorer -- VM som en del av f
    orbundets rekrutteringsstrategi \cite{OsloVM}.
\end{itemize}

\paragraph{PESTEL-analyse}

\begin{table}[ht]
    \centering
    \begin{tabular}{@{}p{2.7cm}p{10.2cm}@{}}
        \toprule                                                                                                                                            \\ \midrule
        \textbf{P}olitiske     & Statlig og kommunal finansiering avhang av at arrangementet leverte brede samfunnsgevinster \cite{TrondheimKommuneVM}.                                                       \\
        \textbf{E}konomiske    & 93~\% av billettene ble solgt nasjonalt; samtidig presset valutarisiko og økte anleggskostnader resultatmarginen \cite{AdressaKjopefest}.                                  \\
        \textbf{S}osiale       & 230\,000 stadiontilskuere og 3\,800 frivillige skapte folkefest, men synkende klubbmedlemskap viser at grasroteffekten ikke er realisert ennå \cite{Adresseavisen,OsloVM}. \\
        \textbf{T}eknologiske  & Prosjektet \textit{Snow for the Future} med automatisert og energieffektiv snøproduksjon gav et eksportbart kompetansefortrinn\cite{Trondheim2025Sustainability}.          \\
        \textbf{E}nvironmental & Klimatiltak som kortreist mat og 10\,000 nyplantede trær reduserte kritikk, men flyreiser fra publikum forble et omdømmerisiko.                                            \\
        \textbf{L}ovmessige    & FIS-reglement, antidopingkrav og offentlige anskaffelsesregler krevde profesjonell prosjektstyring og åpenhet.                                                             \\ \bottomrule
    \end{tabular}
\end{table}

\paragraph{Påvirkning og resultater}
Fra et interessent-perspektiv (pensum kap. 9) er suksess mer enn «on time/on budget»; en eier må 
oppleve \textit{impact success} -- varige gevinster. Ski-VM leverte rekordpublikum, høy TV-rating 
og norsk medaljedominans, noe som styrket NSF både sportslig og omdømmemessig.
Bærekraftsprofilen traff unge målgrupper og kan hjelpe forbundet i kampen om frivillige 
og sponsorer.
Økonomisk ble det rapportert et moderat pluss -- lavere enn håpet, men viktige inntekter vil 
fortsatt komme fra etter-arrangement og markedsrettigheter \cite{Trondheim2025Portal,Adresseavisen}.

\paragraph{Suksessvurdering}
Med tydelige gevinster på tre av fire hovedinteresser kan Ski-VM regnes som en betinget suksess for NSF. Den største svakheten er at medlemstallene foreløpig ikke viser vekst -- et signal om at sosiale effekter må følges opp gjennom bredere grasrot-initiativ og lavterskelarenaer.

\paragraph{Strategisk refleksjon}
Erfaringene fra Trondheim vil trolig prege Skiforbundets strategi:
\begin{itemize}
    \item Økt satsing på bærekraft og teknologi som konkurransefortrinn internasjonalt.
    \item Større vilje til å ta projekteierrollen i partnerskap med kommuner og næringsliv.
    \item En tydeligere publikums- og rekrutteringsstrategi mellom mesterskapene for å omsette VM-effekten til flere aktive medlemmer.
\end{itemize}

Dermed har Ski-VM ikke bare vært en målestokk på NSF sin prosjektkompetanse -- det har også gitt retning for forbundets posisjon i norsk og internasjonal skisport de neste årene.