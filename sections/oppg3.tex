\section{Oppgave 3}
Den norske arbeidslivsmodellen blir kalt en triangelmodell ettersom den er bygd opp av tre aspekter, nemlig økonomisk styring, organisert arbeidsliv og offentlig velferd \cite{Teknologiledelse} (MANGLER SIDETALL). 
Videre kan den norske arbeidslivmodellen bli delt opp i tre nivåer - velferdsstatsmodellen, arbeidslivsmodellen og samarbeidsmodellen. 
Velferdsstatsmodellen handler om ordninger i arbeidslivet, derav arbeidsledighetstrygd, rett til ferie og sykeordninger. 
Dette er et aspekter ved den norske arbeidslivsmodellen som er viktig å ivareta. Dette er fordi ordninger som dette skaper forutsigbarhet i arbeidsmarkedet samt bidrar til økonomisk trygget for arbeidstakere.
 
Innenfor samarbeidsmodellen er gruppe- og teamarbeid svært sentral av den norske modellen. Ifølge BOKA (?SKAL DETTE STÅ SÅNN?) står team sentralt innenfor den norske modellen og er blitt sentralt i norsk arbeidsliv. 
Teamarbeid handler om å utvikle team som får så mye ansvar som mulig gitt de rammebetingelsene man operer innenfor. 
Det å la teamet selv finne gode løsninger er en måte å ivareta tradisjonene i norsk arbeidsliv \cite{Teknologiledelse} (MANGLER SIDETALL), og er derfor et viktig aspekter ved den norske arbeidslivsmodellen som er viktig å ivareta. 
Dette er fordi teamarbeid fremmer selvstendighet og ansvarsfølelse, og når et team team gis frihet til å organisere sitt eget arbeid og finne løsninger innenfor gitte rammer, styrkes både innovasjonsevnen og effektiviteten i organisasjonen.
 
Ifølge en NRK-artikkel publisert januar 2025, er dagens generasjon mer krevende å lede. Den såkalte generasjon Z (Gen Z) mener derimot at det er sjefene som må endre metoder \cite{NRK}. Gen Z vil finne en balanse mellom jobb og fritid, spesielt muligheten til å jobbe mer digitalt og uavhengig av et fast kontor.  Mangelen på fleksibilitet, både når det gjelder arbeidstid og arbeidssted er en utfordring ved dagens arbeidsliv i Norge som bør endres for å bedre tilpasse seg Gen Z sine forventninger. Vi har et godt utgangspunkt for endring siden den norske arbeidslivsmodellen bygger på en triangellmodell mellom arbeidsgivere, arbeidstakere og myndigheter, samt høy grad av tillit og medbestemmelse på arbeidsplassen. Partnerskap mellom tillitsvalgte og ledelse i virksomheten er et viktig moment innenfor den norske modellen \parencite[229]{Teknologiledelse}. 
En mulig løsning er dermed å bruke tillitsvalgte for å inkludere unge ansatte (Gen Z) i beslutninger om fleksible arbeidsordninger. Dette sikrer at endringer skjer i tråd med både arbeidstakers og arbeidsgivers behov.
 
 
Et viktig særtrekk i norsk arbeidsliv er den sentrale rollen som fagforeninger har \parencite[214]{Teknologiledelse}. En annen mulig løsning er dermed å oppdatere tariffavtaler som åpner for større individuell fleksibilitet, for eksempel gjennom avtaler om hjemmekontor uten at det går på bekostning av arbeidsmiljø eller rettigheter. 