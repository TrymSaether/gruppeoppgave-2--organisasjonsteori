\documentclass[a4paper,12pt]{article}
\usepackage{bib/trymtex}

\addbibresource{bib/references.bib}

\begin{document}
\begin{titlepage}
	\newcommand{\HRule}{\rule{\linewidth}{0.5mm}}
	\begin{tikzpicture}[remember picture, overlay]
		% NTNU logo
		\node[anchor=north west, xshift=1.0cm, yshift=-1.0cm] at (current page.north west) {
			\includegraphics[width=2.0cm]{bib/ntnu_logo_liten.png}
		};
	\end{tikzpicture}

	\center

	\vspace*{1cm}

	% Course code & title
	{\color{ntnu-blue}\sffamily\Huge\bfseries TIØ4146 \par}
	\vspace{0.5cm}
	{\sffamily\huge\bfseries Teknologiledelse \par}

	\vspace{1.5cm}

	\HRule
	\vspace{0.5cm}

	% Assignment title
	{\Large\sffamily Gruppeoppgave 2 - Organisasjonsteori\par}
	\vspace{0.3cm}
	%{\Huge\sffamily\textit{EcoBaseNO}\par}

	\vspace{0.5cm}
	\HRule

	\vspace{2cm}

	% Author info
	\begin{minipage}{0.6\textwidth}
		\begin{flushleft}
			\large
			\textbf{Gruppe 186}\\
			10059 \\
			10645 \\
			10813 \\
			10631 \\
			10323 \\
			
		\end{flushleft}
	\end{minipage}%
	\begin{minipage}{0.4\textwidth}
		\begin{flushright}
			\large
			\textbf{Semester:}\\
			Vår 2025
		\end{flushright}
	\end{minipage}

	\vfill

	% University logo/name
	\begin{center}
		{\color{ntnu-blue}\sffamily\Large Norges teknisk-naturvitenskapelige universitet}\\
		\vspace{0.3cm}
		{\sffamily\large Institutt for industriell økonomi og teknologiledelse}

		\vspace{0.5cm}
		{\large 27. Mars 2025}
	\end{center}

	\vspace{1cm}
\end{titlepage}




\setcounter{page}{1}
\renewcommand{\contentsname}{Innholdsfortegnelse}
\tableofcontents
\newpage

\section*{Innledning}

Dette prosjektet er delt inn i tre hovedoppgaver, der alle tre er knyttet til kapitlene 6-10 i pensumboken Teknologiledelse for ingeniørstudier, 2.utgave. 
Oppgavene tar for seg ulike sider ved teknologiledelse og fokuserer på både teori og anvendelse i praksis.

I første hovedoppgave ser på vi på Ski-VM i Trondheim som et prosjekt, der vi skal analysere prosjektet med fokus på blant annet organisering og prosjektledelse. 

Videre, i oppgave 2, er fokuset arbeidslivet og personlig motivasjon for valg av arbeidsgiver. Her skal vi se på en fremtidig arbeidsplass og bruke teori fra pensumboken sammen med offentlig informasjon for å forklare hva ssom gjør denne bedriften attraktiv. 

Den tredje og siste oppgaven tar for seg to aspekter ved den norske arbeidslivsmodellen som bør ivaretas, som utfordringer i norsk arbeidsliv knyttet til Gen Z, dagens generasjon. Oppgaven bygger på aktuell debatt og teori fra kapittel 10 i pensum.

\section{Oppgave 1 - Ski-VM i Trondheim}

Ski-VM i Trondheim er et omfattende arrangement med mange involverte aktører. Det kan kategoriseres som et leveranseprosjekt. Ifølge \parencite[][s.165]{Teknologiledelse} er målet med et leveranseprosjekt å levere en tjeneste eller et produkt. Spesielt kan dette prosjektet betraktes som et arrangement, en ekstrem form for leveranseprosjekt, der målet er å gjennomføre et stort event \parencite[][s.165]{Teknologiledelse}. Ifølge modellen beskrevet i \cite[][s.164]{Teknologiledelse}, kan et prosjekt kjennetegnes av:
\begin{itemize}
    \item \textbf{Tidsavgrensning:} Ski-VM har faste konkurransedatoer som ikke kan utsettes.
    \item \textbf{Ressursrammer:} Materiell, personell og budsjett må være på plass i forkant.
    \item \textbf{Et klart mål:} Det skal gjennomføres et vellykket skiarrangement.
    \item \textbf{Tverrfaglig:} Mange delprosjekter er nødt til å samarbeide for å oppnå et godt arrangement.
\end{itemize}
For et arrangement vil også prosjektorganisasjonen oppløses etter det er ferdig \parencite[][s.165]{Teknologiledelse}.  

Ski-VM involverer både frivillige og betalte ansatte, mange av dem jobber fulltid (eller mer) i gjennomføringsfasen. En hensiktsmessig organisering kan være matriseorganisering, hvor en prosjektleder har overordnet ansvar, og ulike delprosjektledere har ansvar for sine respektive fagområder \parencite[][s.166]{Teknologiledelse}. Denne modellen gir mulighet for tverrfaglig koordinering, samtidig som den legger til rette for spesialisering og effektiv ressursbruk. Matriseorganisering gir fleksibilitet og gjør det lettere å håndtere oppgaver som krysser både fag- og tidsgrenser \parencite[][s.166]{Teknologiledelse}.

Målstyring er et annet sentralt element i prosjektarbeid. Målstyring i prosjekter innebærer at prosjektet er opprettet for å oppnå et klart definert mål, og at dette målet styrer hele prosjektets organisering, gjennomføring og ressursbruk \parencite[][s.164]{Teknologiledelse}. For Ski-VM i Trondheim er det avgjørende at målene er klart definert og godt forankret både hos ledelsen og blant de operative aktørene. Det handler både om prosjektledelsessuksess (at prosjektet følger planen) og prosjektproduktsuksess (at resultatet er vellykket og gir verdi for prosjekteier) \parencite[][s.165]{Teknologiledelse}. Prosjektlederen må forstå hvordan sluttresultatet, et vellykket mesterskap, skaper verdi for interessenter som Norges Skiforbund, FIS og det norske publikummet.

Ettersom Ski-VM er et midlertidig prosjekt med et klart mål innen en fast tidsramme, er tydelig rollefordeling og en effektiv prosjektstruktur essensielt. Dette muliggjør rask kommunikasjon, effektiv beslutningstaking og god risikohåndtering. Dette er spesielt viktig i et prosjekt som involverer både nasjonale og internasjonale aktører, og hvor det kan oppstå raske endringer. God prosjektledelse sikrer at tidsfrister holdes, ressurser brukes effektivt, og målene nås uten store avvik.

\subsection*{Delprosjekter}

\begin{itemize}
    \item \textbf{Logistikk og infrastruktur:} Har ansvar for transport, arenaoppsett og teknisk drift. Dette inkluderer konstruksjon og vedlikehold av løyper og stadion, samt transportløsninger for varer, publikum og utøvere.
    
    \item \textbf{Arrangement og sport:} Ansvarlig for gjennomføringen av konkurransene, tidtaking og seremonier. De samarbeider tett med det internasjonale skiforbundet.
    
    \item \textbf{Sikkerhet og beredskap:} Har ansvar for helseberedskap, sikkerhetsvakter og trafikksikkerhet i arrangementsområdet. De samarbeider tett med politi, brannvesen og helsevesen for å sikre et trygt arrangement.
    
    \item \textbf{Frivillighet og HR:} Ansvarlig for rekruttering, opplæring og oppfølging av frivillige. Frivillige er en bærebjelke i arrangementet, og dette delprosjektet sørger for at de får nødvendig støtte og opplæring.
    
    \item \textbf{Kommunikasjon og media:} Håndterer all ekstern og intern kommunikasjon. De er ansvarlige for kontakt med presse, publisering på sosiale medier og formidling av informasjon til ansatte og frivillige.
    
    \item \textbf{Økonomi og administrasjon:} Følger opp budsjett, regnskap og kontrakter. De har ansvar for at prosjektet gjennomføres innenfor de økonomiske rammene.
\end{itemize}
Følgende er et organisasjonskart som beskriver matrisestrukturen til prosjektet:
\begin{figure}[h]
    \centering
    \includegraphics[width=10cm]{organisasjonskart}
    \caption{Organisasjonskart}
    \label{fig:organisasjonskart}
\end{figure}
\subsection{Var Ski-VM en suksess?}
Det finnes ingen konkret, universalt svar på om Ski-VM i Trondheim er en suksess eller ikke.
I likhet med det meste annet er dette subjektivt, og dermed er det viktigere spørsmålet for hvem
var det en suksess og for hvem var det ikke? For et prosjekt med størrelsen av Ski-VM er det
enormt med aktører og interessenter, derfor velger vi tre som er spesielt interessante,
Trondheim kommune, næringslivet rettet mot Ski-VM og Skiforbundet. Dette er aktører som hadde
spesielt mye innflytelse på-, og var spesielt påvirket av resultatet og konsekvensene av
arrangementet. For å analysere disse gruppene benyttes en PESTEL-analyse\parencite[][s.219]{Teknologiledelse}.

\subsubsection{Trondheim kommune}
Trondheim kommune var den viktigste interessenten av Ski-VM da de klart hadde mest
innflytelse på Ski-VM. Dette er en stor og kompleks gruppe med mange ulike seksjoner som
alle har ulike mål og ønsker til Ski-VM, likevel kan kommunens ønsker inndeles i
følgende hovedmål: P - politisk gevinst, E - økonomisk bærekraft og E - miljømessig bærekraft og troverdighet.

\paragraph{P - Politisk gevinst:} VM styrket Trondheims nasjonale omdømme. Noen dager var det mer enn 40 tusen publikummere i Granåsen\cite{NRKFolkefest}. Statsministerens støtte fremhevet kommunens gjennomføringsevne\cite{Trondheim2025Midler}. Arrangementet ble omtalt som mer et skimesterskap, med særlig vekt på frivillighet og inkludering\cite{Trondheim2025Baerekraft}.  Dette samsvarer med kommunens politikk for sosial inkludering, og bidro til økt lokal stolthet og tillit til kommunens prosjektkapasitet. Undersøkelser viser at majoriteten av de som deltok i undersøkelsen "hadde et positivt totalinntrykk av VM og 69 prosent av dem i aldersgruppen 15-29 år syntes regionen er mer attraktiv etter mesterskapet"\cite{AftenpostenNyVM}.

\paragraph{E - Økonomisk bærekraft:} Trondheim kommune investerte tungt i Granåsen med omlag 1.2 
milliarder kr satt inn i prosjektet\cite{NRK12Milliarder}. Disse enorme summene har blitt møtt med skepsis,
og i ettertid er det ennå ikke tydelig om VM har direkte gått i pluss for Trondheim kommune\cite{AftenpostenNyVM}.
Indirekte kan arrangementet ha vært økonomisk bærekraftig gjennom å stimulere den lokal
økonomien, det er tydelig at det har vært en enormt økt aktivitet i hotell- og
restaurantnæringen\cite{DagbladetPriser}. Kommunen har siden signalisert at fremtidige mesterskap ikke
kan få like mye finansiering fra Trondheim\cite{NeaRadioVM}, noe som kan tyde på at VM har hatt mindre økonomisk
gevinst enn forventet. Samtidig forblir mye av investeringen igjen til gevist for Trondheim i form
et oppgradert, varig skianlegg.

\paragraph{E - Miljømessig troverdighet:} Kommunen og arrangørene hadde mål om et klimanøytralt 
mesterskap\cite{TrondheimKommuneVM}. Blant tiltakene var bruk av utslippsfri energi, grønn 
transport og massiv avfallssortering. Klimagassutslipp ble redusert med 98\% sammenlignet 
med Oslo-VM 2011\cite{Trondheim2025Kutt}. Granåsen ble bygget med miljøhensyn og skal ha lang levetid. 
Kommunens krav presset frem innovative løsninger og styrket Trondheims miljøprofil nasjonalt og internasjonalt.

\subsubsection{Næringslivet rettet mot Ski-VM}
Næringslivet rettet mot Ski-VM var også en viktig interessent. Denne gruppen består av 
hoteller, restauranter, transporttjenester og andre virksomheter som i stor grad hadde sitt 
daglige virke koblet til Ski-VM, enten direkte gjennom tjenester til publikum og deltakere, 
eller indirekte gjennom økt aktivitet i byen. Spesielt viktig innad i denne gruppen er 
næringslivet som ble skapt utelukkende for å ta fordel av Ski-VM, da matbodene og VM-lavvoen 
i sentrum. For næringslivet anslår vi at hovedmålet var å oppnå: P - bygge nettverk, E - økt omsetning,  og S - styrke Trondheim som en attraktiv destinasjon for fremtidige arrangementer og turisme.

\paragraph{P - Politiske vurderinger:}
Det ble politisk debatt om økonomistyring. Kritikere pekte på høyere kostnader enn planlagt, og mente kommunen burde stilt strengere krav\parencite{nettavisenKritikk}. Arrangørene fremhevet på sin side høy sponsorstøtte og publikumsoppslutning som tegn på suksess\parencite{kom24Sponsorsalg}. VM-landsbyen og lavvoen illustrerer hvordan et mesterskap kan forenes med næringsutvikling og byliv.

\paragraph{E - Økonomiske aspekter:}
Næringslivet spilte en nøkkelrolle under Ski-VM i Trondheim. Samtlige sponsorpakker ble solgt, og mange bedrifter opplevde økt omsetning\parencite{kom24Sponsorsalg}. Lokale butikker og serveringssteder, som Rema 1000 ved Granåsen, melder om salgsboom og høy publikumstilstrømning\parencite{nettavisenRema}. Midtbyen utvidet åpningstidene, og lokale aktører som Eggan Nedre tilbød mat, bar og konserter i en stor lavvo\parencite{midtbyenProgram}. Samtidig opplevde enkelte utfordringer, som butikker som fikk utsikten sperret av utedoer midt i handlegata\parencite{nettavisenToalett} Det var også ulik fordeling av inntektene, hvor billigere, uformelle aktører tok noe av salget fra offisielle VM-stand\parencite{nettavisenRema}. I tillegg hadde været en enorm påvirkning på omsetning av VM-bodene. Flere av aktørene rapporterte om store tap både direkte, i form av matsvinn, og potensiell, i form av tapte kunder\parencite{innherredTragedie}.

\paragraph{S - Sosiale effekter:}
Lavvoen og VM-landsbyen ble sentrale for folkefesten i byen, med DJ-er, konserter og 
storskjermer. Enkelte kvelder var det kø for å komme inn i lavvoen, noe som tyder på stor 
interesse. Over 200 tusen besøkte arrangementene på Torvet, med opptil 27 tusen på én kveld\parencite{wikipediaSkiVM}. 
Tilbudene var inkluderende og gratis, og arrangement som “Drømmedagen” for skoleelever bidro til 
sosial bærekraft og lokal stolthet\parencite{wikipediaSkiVM}.

\subsubsection{Norges Skiforbund som interessent}
Skiforbundet er en annen sentral interessent og har hatt stor påvirkning på Ski-VMs utforming
og gjennomføring. Som hovedeier (60 \%) i arrangørselskapet Ski-VM Trondheim 2025 AS og øverste
sportslige myndighet hadde Norges Skiforbund (NSF) en todelt rolle: strategisk prosjekteier med
ansvar for nasjonale mål -- rekruttering, prestasjoner og omdømme -- og kommersiell aktør som
skulle sikre et økonomisk overskudd til videre drift og talentutvikling\cite{ProffSkiVM2025}.

\subsubsection*{Mål og interesser}

\begin{itemize}
    \item \textbf{Økonomi} -- overskudd som kan snu flere år med røde tall og
          medlemstap\cite{Adresseavisen}

    \item \textbf{Sportslig prestasjon} -- norske medaljer som styrker merkevaren Langrenn Norge.

    \item \textbf{Bærekraftig arrangement} som viser forbundets samfunnsansvar \cite{TrondheimKommuneVM}.
    \item \textbf{Omdømmebygging} mot frivillige, ungdom og sponsorer -- VM som en del av f
          orbundets rekrutteringsstrategi \cite{OsloVM}.
\end{itemize}

\subsubsection*{PESTEL-analyse}

\begin{table}[H]
    \centering
    \begin{tabular}{@{}p{2.7cm}p{10.2cm}@{}}
        \toprule                                                                                                                                                                                            \\ \midrule
        \textbf{P}olitiske     & Statlig og kommunal finansiering avhang av at arrangementet leverte brede samfunnsgevinster \cite{TrondheimKommuneVM}.                                                     \\
        \textbf{E}konomiske    & 93~\% av billettene ble solgt nasjonalt; samtidig presset valutarisiko og økte anleggskostnader resultatmarginen \cite{AdressaKjopefest}.                                  \\
        \textbf{S}osiale       & 230\,000 stadiontilskuere og 3\,800 frivillige skapte folkefest, men synkende klubbmedlemskap viser at grasroteffekten ikke er realisert ennå \cite{Adresseavisen,OsloVM}. \\
        \textbf{T}eknologiske  & Prosjektet \textit{Snow for the Future} med automatisert og energieffektiv snøproduksjon gav et eksportbart kompetansefortrinn\cite{Trondheim2025Sustainability}.          \\
        \textbf{E}nvironmental & Klimatiltak som kortreist mat og 10\,000 nyplantede trær reduserte kritikk, men flyreiser fra publikum forble et omdømmerisiko.                                            \\
        \textbf{L}ovmessige    & FIS-reglement, antidopingkrav og offentlige anskaffelsesregler krevde profesjonell prosjektstyring og åpenhet.                                                             \\ \bottomrule
    \end{tabular}
\end{table}

\paragraph{Påvirkning og resultater}
Fra et interessent-perspektiv er suksess mer enn «on time/on budget»; en eier må
oppleve \textit{impact success}, varige gevinster. Ski-VM leverte rekordpublikum, høy TV-rating
og norsk medaljedominans, noe som styrket NSF både sportslig og omdømmemessig.
Bærekraftsprofilen traff unge målgrupper og kan hjelpe forbundet i kampen om frivillige og sponsorer.
Økonomisk ble det rapportert et moderat pluss, lavere enn håpet, men viktige inntekter kan
fortsatt komme fra etter-arrangement og markedsrettigheter, og kunne skapes fra erfaringene akkumulert under Ski-VM\cite{Trondheim2025Portal,Adresseavisen}.

Erfaringene fra Trondheim vil trolig også prege Skiforbundets strategi.
\begin{itemize}
    \item Økt satsing på bærekraft og teknologi som konkurransefortrinn internasjonalt.
    \item Større vilje til å ta projekteierrollen i partnerskap med kommuner og næringsliv.
    \item En tydeligere publikums- og rekrutteringsstrategi mellom mesterskapene for å omsette VM-effekten til flere aktive medlemmer.
\end{itemize}

\subsubsection{Suksessvurdering}
Basert på PESTEL-analysene av de tre hovedinteressentene, kan vi nå vurdere hvorvidt Ski-VM i Trondheim var en suksess. Det er tydelig at svaret varierer avhengig av hvilken interessent vi betrakter og hvilke kriterier som vektlegges.

For Trondheim kommune kan mesterskapet anses som en politisk suksess. Arrangementet styrket byens omdømme nasjonalt og internasjonalt, skapte lokal stolthet og fremhevet kommunens gjennomføringsevne. Miljømessig var mesterskapet også en suksess med 98\% reduksjon i klimagassutslipp sammenlignet med Oslo-VM 2011\cite{Trondheim2025Kutt}. Økonomisk er bildet mer nyansert - de betydelige investeringene (1,2 milliarder kr) har ikke nødvendigvis gitt direkte økonomisk avkastning på kort sikt, men har resultert i varige anlegg som kan gi langsiktig verdi for kommunen.

For næringslivet var resultatene blandede. De fleste aktører opplevde økt omsetning og aktivitet, med særlig suksess for hotell- og restaurantnæringen. Sponsorpakker ble utsolgt, og arrangementer i sentrum tiltrakk over 200 000 besøkende\parencite{wikipediaSkiVM}. Samtidig var det ujevn fordeling av inntektene, og væravhengige aktører opplevde tidvis betydelige tap\parencite{innherredTragedie}. Den sosiale dimensjonen med folkefest og inkluderende arrangementer kan likevel betraktes som en suksess for byens næringsliv.

For Norges Skiforbund leverte mesterskapet flere viktige gevinster: sportslig suksess med norsk medaljedominans, rekordpublikum, høy TV-rating og styrket merkevare. Bærekraftsprofilen traff yngre målgrupper og kan bidra positivt til rekruttering. Økonomisk ble resultatet et moderat overskudd - mindre enn håpet, men likevel positivt for en organisasjon som har slitt med røde tall i flere år\cite{Adresseavisen}.

Samlet sett fremstår Ski-VM 2025 som en betinget suksess. De politiske, sosiale og miljømessige gevinstene var betydelige og overgikk forventningene. De økonomiske resultatene var mer moderate enn forhåpningene tilsa, men likevel akseptable for de fleste involverte. Den varige effekten på rekruttering til skisporten og langsiktig verdiskaping gjenstår å se, og vil være avgjørende for den endelige vurderingen av mesterskapet som en strategisk investering for regionen og skisporten.

Det som særlig fremstår som vellykket var balansen mellom tradisjon og fornyelse: mesterskapet klarte å ivareta skiidrettens historiske og kulturelle betydning, samtidig som det demonstrerte innovasjon innen bærekraft og teknologi. Dette doble fokuset kan tjene som en modell for fremtidige idrettsarrangementer i Norge.

\section{Oppgave 2}
\subsection{Arbeidsgiver og motivasjon}
Twoday er \enquote{(...) en internasjonal leverandør av forretningskritiske IT-løsninger i Norden.}
\parencite{Twoday.no}. Twoday har 2700 ansatte og over 8000 kunder. De jobber med: Data og AI, Software Enginering og Digital Experiences. Selskapet tilbyr variasjon i mulige arbeidsoppgaver, og nettsiden viser at de
verdsetter både generalister og spesialister. Tilpasningsevne og drivkraft er kjerneverdier hos twoday. Twoday er
en arbeidsgiver vi ser på som attraktiv, og vi vil videre diskutere hvordan deres fremstilling som seg selv som 
arbeidsgiver er motiverende på oss med grunnlag i pensumlitteraturen. 

Teorien om psykologiske jobbkrav vektlegger læring og variasjon presenteres som krav for å være tilfreds på jobb \parencite[][s.120]{Teknologiledelse}. 
Kjerneverdiene knyttet til tilpasningsevne og drivkraft er i lys av dette svært viktige for å skape motivasjon hos ansatte. Dette er også motiverende for 
oss da vi ønsker en arbeidsgiver hvor vi kan fortsette å utvikle oss og lære slik vi er vant til fra studiehverdagen.
Satsing på personlig utvikling og muligheten til å bestemme selv står sentralt hos twoday: \enquote{Du har muligheten
til å påvirke egen hverdag, og får en bratt læringskurve og mange utfordringer. Vår ambisjon er å
skape stolte øyeblikk(...).} \parencite["Graduateprogrammet"]{Twoday.no}. I tillegg til å vektlegge læring, ser
vi her at twoday vektlegger autonomi. Som beskrevet i Self Determination Theory\parencite[][s.121]{Teknologiledelse},
autonomi og å jobbe av egen fri vilje sentralt for å skape indre motivasjon hos mennesker. Læring,
variasjon og å bestemme over eget arbeid er sentralt for selvrealisering-toppen av Maslows
behovspyramide \parencite[][s.116]{Teknologiledelse}.

Twoday prioriterer mennesker, samhold og bærekraft:  
«Vi bryr oss om våre ansatte, kunder og samfunnet …» \parencite["Om oss"]{Twoday.no}.  
Slike verdier møter både sosiale behov i Maslows pyramide og kravet “sammenheng unngår
fremmedgjøring” fra teorien om psykologiske jobbkrav \parencite[][s.120]{Teknologiledelse}. i jobbkrav-teorien \parencite[][s. 120]{Teknologiledelse}.  
I tillegg ønsker selskapet å gjøre hverdagen enklere for folk \parencite["Graduateprogrammet"]{Twoday.no}, noe som appellerer til vår motivasjon om å bidra til noe større.  



Med vekt på motiverende faktorer som sosialt samhold, selvutvikling, læring, mestring og utfordringer er derfor
twoday en motiverende arbeidsgiver. Flere sentrale motiverende faktorer fra litteraturen, som vi også subjektivt verdsetter
er tydelig i fokus hos dem, noe som gjør dem til en motiverende og interessant arbeidsgiver, \parencite["Graduateprogrammet"]{Twoday.no}. Dette er motiverende for oss, da vi ønsker å delta i noe
større, og utgjøre en forskjell med den jobben vi gjør.




\section{Oppgave 3}
Den norske arbeidslivsmodellen blir kalt en triangelmodell ettersom den er bygd opp av tre aspekter, nemlig økonomisk styring, organisert arbeidsliv og offentlig velferd \parencite{Teknologiledelse} (s.214). 
Videre kan den norske arbeidslivmodellen bli delt opp i tre nivåer - velferdsstatsmodellen, arbeidslivsmodellen og samarbeidsmodellen. 
Velferdsstatsmodellen handler om ordninger i arbeidslivet, derav arbeidsledighetstrygd, rett til ferie og sykeordninger. 
Dette er et aspekter ved den norske arbeidslivsmodellen som er viktig å ivareta. Disse ordninger skaper forutsigbarhet i arbeidsmarkedet samt bidrar til økonomisk trygget for arbeidstakere.
 
Innenfor samarbeidsmodellen er gruppe- og teamarbeid svært sentral av den norske modellen. 
Teamarbeid handler om å utvikle team som får så mye ansvar som mulig gitt de rammebetingelsene man opererer innenfor, og står team sentralt innenfor den norske modellen og er blitt sentralt i norsk arbeidsliv. 
Det å la teamet selv finne gode løsninger er en måte å ivareta tradisjonene i norsk arbeidsliv \parencite{Teknologiledelse} (s.277), og er derfor et aspektet ved den norske arbeidslivsmodellen som er viktig å ivareta. 
Dette er fordi teamarbeid fremmer selvstendighet og ansvarsfølelse, og med friheten til å organisere sitt eget arbeid og finne løsninger innenfor gitte rammer, styrkes både innovasjonsevnen og effektiviteten i organisasjonen.
 
Ifølge en NRK-artikkel publisert januar 2025, er dagens generasjon mer krevende å lede. Den såkalte generasjon Z (Gen Z) mener derimot at det er sjefene som må endre metoder \parencite{NRK}. Gen Z vil finne en balanse mellom jobb og fritid, spesielt muligheten til å jobbe mer digitalt og uavhengig av et fast kontor.  Mangelen på fleksibilitet, både når det gjelder arbeidstid og arbeidssted er en utfordring ved dagens arbeidsliv i Norge.  Vi har et godt utgangspunkt for endring siden den norske arbeidslivsmodellen bygger på en triangellmodell mellom arbeidsgivere, arbeidstakere og myndigheter, samt høy grad av tillit og medbestemmelse på arbeidsplassen. Partnerskap mellom tillitsvalgte og ledelse i virksomheten er et viktig moment innenfor den norske modellen \parencite[229]{Teknologiledelse}. 
En mulig løsning er dermed å bruke tillitsvalgte for å inkludere unge ansatte (Gen Z) i beslutninger om fleksible arbeidsordninger.
 
 
Et viktig særtrekk i norsk arbeidsliv er den sentrale rollen som fagforeninger har \parencite[214]{Teknologiledelse}. En annen mulig løsning er dermed å oppdatere tariffavtaler som åpner for større individuell fleksibilitet, for eksempel gjennom avtaler om hjemmekontor uten at det går på bekostning av arbeidsmiljø eller rettigheter. 

\section*{Konklusion}
Målet med dette prosjektet er å få en helhetlig innsikt i sentrale teamer innen teknologiledelse, prosjektarbeid og organisasjonsutvkling. 
Vi har anvendt teorier og modeller fra pensum for å analysere et konkret prosjekt, nemlig Ski-VM i Trondheim, og også utforsket perosnlige motivasjoner for karrierevalg samt studere hvordan det norske arbeidslivet kan utviklet i møte med Gen Z. 
Disse tre hovedoppgavene kobler teori og praksis sammen, og har gitt oss en bedre forståelse av både prosjektledelse og strategisk tenkning rundt fremtidens arbeidsliv. Gjennom dette prosjektet har vi styrket våre evner til kritisk refleksjon, både prosjektledelse og strategisk tenkning rundt fremtidens arbeidsliv. 

\printbibliography[heading=bibintoc, title={Referanser}, notcategory=others]

\end{document}